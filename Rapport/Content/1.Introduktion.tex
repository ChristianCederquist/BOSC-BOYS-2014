\chapter{Introduktion}
Denne rapport undersøger Linux's kernemoduler og processer. Såkaldte Loadable Kernel Modules tillader en bruger at installere nye moduler i kernen uden at skulle genstarte og genopbygge operativsystemet. Vi bruger her disse kernemoduler til at udføre simple opgaver, så som printing af informationer i structs, og til at gennemgå systemets liste af processer, både ved lineær gennemgang og ved dybde-først søgning. Vi anvender de forskellige macros i de tilgængelige biblioteker til liste-gennemgang. Vi diskuterer til sidst fordele og ulemper ved kernemoduler, indbyggede kommandoer til at se processer og en sammenligning af førnævnte med vores implementation, samt ændringsforslag og muligheder for udvidelse eller optimeringer af kernemodulet. Vi konkluderer, at det nuværende modul ikke er lige så godt som ps-kommandoen, men at der er mange muligheder for at udvide og videreudvikle vores kernemodul. 