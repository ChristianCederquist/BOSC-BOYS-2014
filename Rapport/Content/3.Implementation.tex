\chapter{Implementation}
\section{Simple}
Simple er et modul som bruges til at lave en linked liste ud fra en defineret datastruktur. Når modulet bliver initialiseret, skal den oprette en liste og udkrive alle elementer fra den den. Når modulet bliver fjernet skal den slette hele listen og frigøre pladsen.
\\\\
Simple er implementeret ved først at definere en struct som indeholder de fornødne data, som man gerne vil have til at indgå i den linkede liste. Det særlige ved structén er at den indeholder: \textit{struct list\_head list}. \textit{list\_head} er en del af \textit{linux{types.h}} package. Den indeholder 2 members: next og prev, som henholdsvis peger på den næste og det tidligere element i listen. Denne specifikke struktur gør det muligt at anvende macro-funktioner til at behandle den definerede datastruktur.
\\\\
For at indsætte elementer i listen anvendes en macro, som kan ses i kodeudsnittet nedenunder. 

\begin{lstlisting}
list_add_tail(&person->list, &birthday_list);
\end{lstlisting}

For at iterere igennem listen og udskrive de forskellige elementer, anvendes macroén: \textit{list\_for\_each\_entry}.

\begin{lstlisting}
list_for_each_entry(ptr, &birthday_list, list){
         printk(KERN_INFO "Person %d-%d-%d\n", ptr->day, ptr->month, ptr->year);
       }
\end{lstlisting}

For at slette elementer fra listen anvendes 2 Macroér: \textit{list\_for\_each\_entry\_safe} som bruges til at iterere igennem listen, og \textit{list\_del} som anvendes til at slette elementer. Derudover anvendes også metoden kfree, som frigøre den anvendte hukommelse.

\begin{lstlisting}
list_for_each_entry_safe(ptr, next, &birthday_list, list){
	  list_del(&ptr->list);
	  kfree(ptr);
	}
\end{lstlisting}

\section{Linear}
Dette modul har til opgave at udskrive alle tasks som Linux udfører på et pågældende tidspunkt. 

I Linux er alle task repræsenterert ved hjælp af structen: \textit{task\_struct}. Derfor er implementationen af dette modul meget simpel, da den kun har brug for at iterere igennem listen af tasks, som den får givet af systemet og derefter printe de fornødne informationer ud. For at udføre iterationen anvendes macroén: \textit{for\_each\_process}.

\begin{lstlisting}
for_each_process(task){
	  printk(KERN_INFO "%s: State is %ld and process id is %d\n",task->comm,task->state,task->pid);
	}
\end{lstlisting}

\section{DFS}
Dette modul skal ligesom det foregående modul, udskrive alle tasks som Linux er igang med at udføre. Foreskellen er dog at modulet ikke skal udskrive tasks linært, men derimod anvende Depth-Fisrt Search Tree.
\\
\begin{lstlisting}
int dfs_list(struct task_struct *task){
	struct list_head *list;	  
	struct task_struct *child;

  	list_for_each(list, &task->children){
	  child = list_entry(list, struct task_struct, sibling);	
	  printk(KERN_INFO "%s: State is %ld and process id is %d\n",child->comm,child->state,child->pid);
	  dfs_list(child);
        }
  return 0;
}
\end{lstlisting}